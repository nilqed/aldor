\thistype{UnivariatePolynomialDiophantineSolver}
\History{Manuel Bronstein}{29/7/2004}{created}
\Usage{import from \this(R, RX)}
\Params{
\emph{R} & \alexttype{IntegralDomain}{} & the coefficient ring\\
\emph{RX}
& \alexttype{UnivariatePolynomialAlgebra}{} R & polynomials over {\em R}\\
}
\Descr{\this(R, RX) implements diophantine equation solving in \emph{RX}.}
\begin{exports}
\alexp{pseudoDiophantine}: &
(RX, RX, RX) $\to$ \alexttype{Partial}{} \builtin{Cross}(R, RX, RX) &
equation solver\\
\end{exports}
\alpage{pseudoDiophantine}
\Usage{\name(a, b, c)}
\Signature{(RX, RX, RX)}{\alexttype{Partial}{} \builtin{Cross}(R, RX, RX)}
\Params{ {\em a,b,c} & RX & Polynomials\\ }
\Retval{
Returns either $(r, x, y)$ such that $a x + b y = r c$ and
$x=0$ or $\deg(x) < \deg(b)$, or \failed.}
\Remarks{If \name{} returns \failed,
then the equation $a x + b y = r c$ has no solution
whenever \emph{R} is a field or $\gcd(a,b) \in R$.
If those two conditions are not met, then
a solution could have been missed.
Therefore, \name{} is
complete only when \emph{R} is a field or when
\emph{a} and \emph{b} have no common root.}
