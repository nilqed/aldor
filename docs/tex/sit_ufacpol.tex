\thistype{UnivariateFactorialPolynomial}
\History{Manuel Bronstein}{16/6/2000}{created}
\Usage{ import from \this(R, Rx)}
\Params{
{\em R} & \altype{Ring} & The coefficient domain\\
{\em Rx} & \altype{UnivariatePolynomialAlgebra} R & A polynomial type over R\\
}
\Descr{\this(R, Rx) implements univariate factorial polynomials with
coefficients in $R$.
Those are polynomials with respect to the basis
of the descending factorials $(x^{{\underline n}})_{n \ge 0}$,
where $x^{{\underline n}} = x (x-1) \dots (x-n+1)$.
Rx is used for representing
the factorial polynomials, so you can choose between sparse and dense
representations.}
\begin{exports}
\category{\altype{UnivariateFreeRing} R}\\
\alexp{coerce}: & Rx $\to$ \% & Conversion to a factorial polynomial\\
\alexp{expand}: & \% $\to$ Rx & Conversion from a factorial polynomial\\
\alexp{trailExpand}:
& \% $\to$ (\altype{Integer}, Rx) & Conversion from a factorial polynomial\\
\end{exports}
\begin{exports}[if $R$ has \altype{CommutativeRing} then]
\category{\altype{CommutativeRing}}\\
\end{exports}
\begin{exports}[if $R$ has \altype{IntegralDomain} then]
\category{\altype{IntegralDomain}}\\
\end{exports}
\begin{exports}[if $R$ has \altype{RationalRootRing} then]
\alexp{integerRoots}:
& \% $\to$ \altype{List} \altype{FractionalRoot} \altype{Integer} &
Integer roots\\
\alexp{rationalRoots}:
& \% $\to$ \altype{List} \altype{FractionalRoot} \altype{Integer} &
Rational roots\\
\end{exports}
\alpage{coerce,expand,trailExpand}
\altarget{coerce}
\altarget{expand}
\altarget{trailExpand}
\Usage{p::\%\\ coerce~p\\ expand~q\\ (n, h) := trailExpand~q}
\Signatures{
coerce: & Rx $\to$ \%\\
expand: & \% $\to$ Rx\\
trailExpand: & \% $\to$ (\altype{Integer}, Rx)\\
}
\Params{
{\em p} & Rx & A polynomial\\
{\em q} & \% & A factorial polynomial\\
}
\Descr{p::\% converts {\em p} from the power basis $(x^n)_{n \ge 0}$ to
the factorial basis $(x^{{\underline n}})_{n \ge 0}$, while expand(q)
performs the reverse conversion and trailExpand(q) returns $(n, h)$
such that $q = x^{{\underline n}} h$.}
\alpage{integerRoots,rationalRoots}
\altarget{integerRoots}
\altarget{rationalRoots}
\Usage{ integerRoots~p\\ rationalRoots~p }
\Signature{\%}{\altype{List} \altype{FractionalRoot} \altype{Integer}}
\Params{ {\em p} & \% & A factorial polynomial\\ }
\Retval{Return $[(r_1,e_1),\dots,(r_n,e_n)]$ where the $r_i$'s are
the integer or rational roots of $p$ and have multiplicity $e_i$.}
