\thistype{Automorphism}
\History{Manuel Bronstein}{4/11/94}{created}
\Usage{import from \this~R}
\Params{
{\em R} & \astype{Ring} & The ring on which the automorphisms operate\\
}
\Descr{\this~R provides automorphims on $R$.}
\begin{exports}
\category{\astype{Group}}\\
\asexp{apply}:
& (\%, R, \astype{Integer}) $\to$ R & Apply a morphism to an element of $R$\\
\asexp{function}: & \% $\to$ (R $\to$ R) & Action of a morphism\\
\asexp{morphism}: & (R $\to$ R) $\to$ \% & Create a morphism\\
                  & (R $\to$ R, R $\to$ R) $\to$ \% & \\
                  & ((R, \astype{Integer}) $\to$ R) $\to$ \% & \\
\end{exports}
\aspage{apply}
\Usage{ $\sigma x$\\ $\sigma(x, n)$\\
\name($\sigma$, x)\\ \name($\sigma$, x, n) }
\Signature{(\%, R, n:\astype{Integer} == 1)}{R}
\Params{
$\sigma$ & \% & An automorphism of $R$\\
{\em x} & R & An element of $R$\\
{\em n} & \astype{Integer} & The number of times to apply (optional)\\
}
\Retval{Returns $\sigma^n x$.}
\aspage{function}
\Usage{\name~$\sigma$}
\Signature{\%}{(R $\to$ R)}
\Params{ $\sigma$ & \% & An automorphism of $R$\\ }
\Retval{Returns the map corresponding to the action of $\sigma$ on the ring.}
\aspage{morphism}
\Usage{ \name~f\\ \name($f, f^{-1}$)\\ \name~g }
\Signatures{
\name: & (R $\to$ R) $\to$ \%\\
\name: & (R $\to$ R, R $\to$ R) $\to$ \%\\
\name: & ((R, \astype{Integer}) $\to$ R) $\to$ \%\\
}
\Params{
{\em f} & R $\to$ R & A function\\
$f^{-1}$ & R $\to$ R & The inverse function of $f$\\
{\em g} & (R, \astype{Integer}) $\to$ R & A function\\
}
\Descr{
\name~f creates the morphism $\sigma$ on $R$ given by
$$
\sigma x = f(x)
$$
for any $x \in R$. The morphism is not necessarily invertible, so any attempt
to use its inverse causes an error.\\
\name($f, f^{-1}$) creates the invertible morphism $\sigma$ on $R$ given by
$$
\sigma x = f(x)\qquad \sigma^{-1} x = f^{-1}(x)
$$
for any $x \in R$.\\
\name~g creates the morphism $\sigma$ on $R$ given by
$$
\sigma^n x = g(x, n)
$$
for any $x \in R$. This morphism is considered invertible, so $g$ must also
be defined for negative integers.}
\Remarks{The maps passed as arguments must be ring morphisms, and the
maps $f$ and $f^{-1}$ must be inverses of each other. When an efficient
algorithm for computing $\sigma^n$ is known, for example for $\sigma = 1_R$,
then the form \name~g with g:~(R, Integer) $\to$ R should be used to
avoid repeated iterations of $\sigma$, which is the default behavior.}
