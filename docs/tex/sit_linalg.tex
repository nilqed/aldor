\thistype{LinearAlgebra}
\History{Manuel Bronstein}{25/11/99}{created}
\Usage{import from \this(R, M)}
\Params{
{\em R} & \altype{CommutativeRing} & The coefficient domain\\
{\em M} & \altype{MatrixCategory} R & A matrix type\\
}
\Descr{\this(R, M) provides basic linear algebra functionalities
for matrices over $R$.}
\begin{exports}
\alexp{invertibleSubmatrix}:
& M $\to$ (\altype{Boolean}, AZ, AZ) & Probable maximal minor\\
\alexp{maxInvertibleSubmatrix}: & M $\to$ (AZ, AZ) & Maximal minor\\
\alexp{span}: & M $\to$ AZ & Span\\
\end{exports}
\begin{exports}[if R has \altype{IntegralDomain} then]
\alexp{cycle}: & (V $\to$ V, V) $\to$ (V, M) &  First dependence relation\\
\alexp{cycle}: & (M, V) $\to$ (V, M) & First dependence relation\\
\alexp{determinant}: & M $\to$ R & Determinant\\
\alexp{factorOfDeterminant}:
& M $\to$ (\altype{Boolean}, R) & Probable determinant\\
\alexp{firstDependence}:
& (\altype{Generator} V, \altype{MachineInteger}) $\to$ V &
First dependence relation\\
\alexp{inverse}: & M $\to$ (M, V) & Inverse\\
\alexp{kernel}: & M $\to$ M & Kernel\\
\alexp{particularSolution}: & (M, M) $\to$ (M, V) & A solution\\
\alexp{rank}: & M $\to$ \altype{MachineInteger} & Rank\\
\alexp{rankLowerBound}:
& M $\to$ (\altype{Boolean}, \altype{MachineInteger}) & Probable rank\\
\alexp{solve}: & (M, M) $\to$ (M, M, V) & All solutions\\
\alexp{subKernel}: & M $\to$ (\altype{Boolean}, M) & Subspace of the kernel\\
\end{exports}
\begin{aswhere}
AZ &==& \altype{Array} \altype{MachineInteger}\\
V &==& \altype{Vector} R\\
\end{aswhere}
\alpage{cycle}
\Usage{\name(f,v)\\ \name(A,v)}
\Signatures{
\name: & (\altype{Vector} R $\to$ \altype{Vector} R, \altype{Vector} R)
$\to$ (\altype{Vector} R, M)\\
\name: & (M, \altype{Vector} R) $\to$ (\altype{Vector} R, M)\\
}
\Params{
{\em f} & \altype{Vector} R $\to$ \altype{Vector} R & A map\\
{\em A} & M & A matrix\\
{\em v} & \altype{Vector} R & A vector whose cycle is wanted\\
}
\Retval{
Returns $([a_0,\dots,a_n], m)$ where
$$
\sum_{i=0}^n a_i A^i v = 0
\quad\paren{\mbox{resp.~} \sum_{i=0}^n a_i f^i(v) = 0}\,,
$$
and the columns of $m$ are $v,f(v),\dots,f^n(v)$
(resp.~$v,Av,\dots,A^n v$).
}
\Remarks{
The relation is as small as possible, meaning that
$v,f(v),\dots,f^{n-1}(v)$
(resp.~$v,Av,\dots,A^{n-1} v$)
are linearly independent over R.
The iterates of $v$ under $f$ must all have the same dimension.
}
\alseealso{\alexp{firstDependence}}
\alpage{determinant}
\Usage{\name~a}
\Signature{M}{R}
\Params{ {\em a} & M & A matrix\\ }
\Retval{ Returns the determinant of {\em a}.}
\alseealso{\alexp{factorOfDeterminant}}
\alpage{factorOfDeterminant}
\Usage{\name~a}
\Signature{M}{(\altype{Boolean}, R)}
\Params{ {\em a} & M & A matrix\\ }
\Retval{ Returns $(det?, d)$ such that $d$ is always a factor of
the determinant of {\em a}, and $d$ is exactly the determinant of $a$
if $det?$ is \true. }
\Remarks{$d$ can also happen to be the determinant of $a$ when $det?$ is
\false, but the algorithm was unable to prove it.}
\alseealso{\alexp{determinant}}
\alpage{firstDependence}
\Usage{\name(gen,n)}
\Signature{(\altype{Generator} \altype{Vector} R, \altype{MachineInteger})}
{\altype{Vector} R}
\Params{
{\em gen} & \altype{Generator} \altype{Vector} R & A generator of vectors\\
{\em n} & \altype{MachineInteger} & The dimension of the vectors generated\\
}
\Descr{
Returns a vector {\em v} which contains the coefficients
of a dependence relation among the vectors generated by {\em gen}. The
relation is as small as possible, meaning that if {\em v} has
dimension {\em m} then the first $m-1$ vectors generated are
linearly independent over R.
The dimension of the vectors generated by {\em gen} must be
{\em n}. There must be a relation between the vectors generated.
}
\alseealso{\alexp{cycle}}
\alpage{inverse}
\Usage{\name~a}
\Signature{M}{(M, \altype{Vector} R)}
\Params{ {\em a} & M & A matrix\\ }
\Retval{ Returns $(b, [d_1,\dots,d_n])$ such that
$$
a b = \pmatrix{
d_1 &     &        & \cr
    & d_2 &        & \cr
    &     & \ddots & \cr
    &     &        & d_n \cr
}\,.
$$
}
\Remarks{$\prod_{i=1}^n d_i \ne 0$ if and only if $a$ is invertible.
In that case, $a^{-1} = b d^{-1}$ where $d$ is the diagonal matrix
with diagonal $d_1,\dots,d_n$. To compute the inverse of $a$ as
a product of a diagonal matrix on the left rather than the right,
let $(b,d)$ be the result of calling
\name{} on \alfunc{MatrixCategory}{transpose}($a$).
Then, $(a^t)^{-1} = b d^{-1}$, so $a^{-1} = d^{-1} b^t$.
When $R$ is a \altype{Field}, then $d_i \in \{0,1\}$ for $1 \le i \le n$,
so $b = a^{-1}$ when $R$ is a \altype{Field} and $a$ is invertible.}
\alpage{invertibleSubmatrix}
\Usage{\name~a}
\Signature{M}{(\altype{Boolean}, \altype{Array} \altype{MachineInteger},
\altype{Array} \altype{MachineInteger})}
\Params{ {\em a} & M & A matrix\\ }
\Retval{ Returns $(max?, [r_1,\dots,r_r], [c_1,\dots,c_r])$
where $r \le \mbox{rank}(a)$
and the submatrix of $a$ formed by the intersections of the rows $r_i$
and $c_i$ is always invertible. If $max?$ is \true, then $r$ is exactly
the rank of $a$ and the given minor is of maximal size.}
\Remarks{$r$ can also happen to be the rank of $a$ when $max?$ is \false,
but the algorithm was unable to prove it.}
\alseealso{\alexp{maxInvertibleSubmatrix}}
\alpage{kernel}
\Usage{\name~a}
\Signature{M}{M}
\Params{ {\em a} & M & A matrix\\ }
\Retval{ Returns a matrix whose columns form a basis of the kernel of {\em a}.}
\alseealso{\alexp{solve},\alexp{subKernel}}
\alpage{maxInvertibleSubmatrix}
\Usage{\name~a}
\Signature{M}{(\altype{Array} \altype{MachineInteger},
\altype{Array} \altype{MachineInteger})}
\Params{ {\em a} & M & A matrix\\ }
\Retval{ Returns $([r_1,\dots,r_r], [c_1,\dots,c_r])$
where $r$ is the rank of {\em a}
and the submatrix of $a$ formed by the intersections of the rows $r_i$
and $c_i$ is invertible.}
\alseealso{\alexp{invertibleSubmatrix}}
\alpage{particularSolution}
\Usage{\name(a, b)}
\Signature{(M, M)}{(M, \altype{Vector} R)}
\Params{ {\em a, b} & M & Matrices\\ }
\Retval{ Returns $(m, [d_1,\dots,d_n])$ such that
$$
a m = b \pmatrix{
d_1 &     &        & \cr
    & d_2 &        & \cr
    &     & \ddots & \cr
    &     &        & d_n \cr
}\,.
$$
}
\Remarks{For each $i$, $d_i \ne 0$ if and only if the system
$a x = \sth{i}$ column of $b$ has a solution, which is then $d_i^{-1}$
times the $\sth{i}$ column of $m$.
When $R$ is a \altype{Field}, then $d_i \in \{0,1\}$ for $1 \le i \le n$,
so $m$ is a solution of $a x = b$ when $R$ is a \altype{Field} and all
the $d_i$'s are nonzero.}
\alseealso{\alexp{solve}}
\alpage{rank}
\Usage{\name~a}
\Signature{M}{\altype{MachineInteger}}
\Params{ {\em a} & M & A matrix\\ }
\Retval{ Returns the rank of {\em a}.  }
\alseealso{\alexp{rankLowerBound},\alexp{span}}
\alpage{rankLowerBound}
\Usage{\name~a}
\Signature{M}{(\altype{Boolean}, \altype{MachineInteger})}
\Params{ {\em a} & M & A matrix\\ }
\Retval{ Returns $(rank?, r)$ such that $r \le \mbox{rank}(a)$,
and $r$ is exactly the rank of $a$ if $rank?$ is \true. }
\Remarks{$r$ can also happen to be the rank of $a$ when $rank?$ is \false,
but the algorithm was unable to prove it.}
\alseealso{\alexp{rank},\alexp{span}}
\alpage{span}
\Usage{\name~a}
\Signature{M}{\altype{Array} \altype{MachineInteger}}
\Params{ {\em a} & M & A matrix\\ }
\Retval{ Returns $[c_1,\dots,c_r]$ where $r$ is the rank of {\em a}
and the span of $a$ is generated by its columns $c_1,\dots,c_r$.}
\alseealso{\alexp{rank}}
\alpage{solve}
\Usage{\name(a, b)}
\Signature{(M, M)}{(M, M, \altype{Vector} R)}
\Params{ {\em a, b} & M & Matrices\\ }
\Retval{ Returns $(w, m, [d_1,\dots,d_n])$ such that the columns
of $w$ form a basis of the kernel of $a$ and
$$
a m = b \pmatrix{
d_1 &     &        & \cr
    & d_2 &        & \cr
    &     & \ddots & \cr
    &     &        & d_n \cr
}\,.
$$
}
\Remarks{For each $i$, $d_i \ne 0$ if and only if the system
$a x = \sth{i}$ column of $b$ has a solution, which is then $d_i^{-1}$
times the $\sth{i}$ column of $m$.
When $R$ is a \altype{Field}, then $d_i \in \{0,1\}$ for $1 \le i \le n$,
so the general solution of $a x = b$ when $R$ is a \altype{Field} and all
the $d_i$'s are nonzero is $x = m + \sum_j r_j w_j$ where $w_j$ is the
$\sth{j}$ column of $w$.}
\alseealso{\alexp{kernel},\alexp{particularSolution}}
\alpage{subKernel}
\Usage{\name~a}
\Signature{M}{(\altype{Boolean}, M)}
\Params{ {\em a} & M & A matrix\\ }
\Retval{ Returns $(ker?, m)$ such that the columns of $m$, which are always
linearly independent over R, generate a subspace of the kernel of $a$,
and generate the full kernel if $ker?$ is \true.}
\Remarks{$m$ can also happen to generate the full kernel of $a$ when $ker?$ is
\false, but the algorithm was unable to prove it.}
\alseealso{\alexp{kernel}}
