\thistype{FractionalRoot}
\History{Manuel Bronstein}{11/12/96}{created RationalRoot}
\History{Manuel Bronstein}{17/8/2000}{added the parameter R, changed type name}
\Usage{import from \this}
\Descr{\this(R) provides fractions of R with multiplicities.}
\Params{ {\em R} & \altype{CommutativeRing} & A ring\\ }
\begin{exports}
\category{\altype{ExpressionType}}\\
\alexp{fractionalRoot}: & (R, R, \altype{Integer}) $\to$ \% & Create a root\\
\alexp{integral?}:
& \% $\to$ \altype{Boolean} & Test whether root is integral\\
\alexp{integralRoot}: & (R, \altype{Integer}) $\to$ \% & Create a root\\
\alexp{integralValue}: & \% $\to$ R & Value of an integral root\\
\alexp{multiplicity}: & \% $\to$ \altype{Integer} & Multiplicity of a root\\
\alexp{setMultiplicity!}:
& (\%, \altype{Integer}) $\to$ \% & Change a multiplicity\\
\alexp{value}: &  \% $\to$ (R, R) & Value of a root\\
\end{exports}
\alpage{integral?}
\Signature{\%}{\altype{Boolean}}
\Usage{\name~r}
\Params{ {\em r} & \% & A root\\ }
\Retval{Return \true~if r is in R, \false~otherwise.}
\alpage{fractionalRoot,integralRoot}
\altarget{fractionalRoot}
\altarget{integralRoot}
\Usage{fractionalRoot(a, b, n)\\ integralRoot(a, n)}
\Signatures{
fractionalRoot: & (R, R, \altype{Integer}) $\to$ \%\\
integralRoot: & (R, \altype{Integer}) $\to$ \%\\
}
\Params{
{\em a} & R & A numerator\\
{\em b} & R & A denominator\\
{\em n} & \altype{Integer} & A multiplicity\\
}
\Retval{Return the root $a$ or $a/b$ with multiplicity $n$.}
\alpage{integralValue}
\Usage{ \name~r }
\Signature{\%}{\altype{Integer}}
\Params{ {\em r} & \% & A root\\ }
\Retval{Returns the value of the integral root $r$, ignoring its multiplicity.}
\alseealso{\alexp{value}}
\alpage{multiplicity}
\Usage{ \name~r }
\Signature{\%}{\altype{Integer}}
\Params{ {\em r} & \% & A root\\ }
\Retval{Return the multiplicity of $r$.}
\alpage{setMultiplicity!}
\Usage{ \name(r, m) }
\Signature{(\%, \altype{Integer})}{\altype{Integer}}
\Params{
{\em r} & \% & A root\\
{\em m} & \altype{Integer} & Its new multiplicity\\
}
\Descr{Sets the multiplicity of $r$ to $m$ and returns $r$.}
\alpage{value}
\Usage{ (n, d) := \name~r }
\Signature{\%}{(\altype{Integer}, \altype{Integer})}
\Params{ {\em r} & \% & A root\\ }
\Retval{Return $(n, d)$ such that the value of $r$ is $n/d$.}
\alseealso{\alexp{integralValue}}
