\thistype{EuclideanDomain}
\History{Manuel Bronstein}{13/12/94}{created}
\Usage{\this: Category}
\Descr{\this~is the category of commutative Euclidean domains.}
\begin{exports}
\category{\altype{GcdDomain}}\\
\alexp{diophantine}:
& (\%, \%, \%) $\to \partial$ \% & Linear diophantine solver\\
\alexp{divide}: & (\%, \%) $\to$ (\%, \%) & Euclidean division\\
\alexp{divide!}: & (\%, \%, \%) $\to$ (\%, \%) & In--place Euclidean division\\
\alexp{euclid}: & (\%, \%) $\to$ \% & Euclidean gcd\\
\alexp{euclid!}: & (\%, \%) $\to$ \% & In--place Euclidean gcd\\
\alexp{euclideanSize}:
& \% $\to$ \altype{Integer} & Size function of the domain\\
\alexp{extendedEuclidean}:
& (\%, \%) $\to$ (\%, \%, \%) & Extended Euclidean Algorithm\\
& (\%, \%, \%) $\to \partial$ \builtin{Cross}(\%, \%) &\\
\alexp{quo}: & (\%, \%) $\to$ \% & Quotient\\
\alexp{rem}: & (\%, \%) $\to$ \% & Remainder\\
\alexp{remainder!}: & (\%, \%) $\to$ \% & In--place remainder\\
\alexp{rationalReconstruction}:
& (\%, \%, Z, Z) $\to \partial$ \builtin{Cross}(\%, \%) &
Rational reconstruction\\
\end{exports}
\begin{alwhere}
$\partial$ &==& \altype{Partial}\\
Z &==& \altype{Integer}\\
\end{alwhere}
\alpage{diophantine}
\Usage{\name(a, b, m)}
\Signature{(\%, \%, \%)}{\altype{Partial} \%}
\Params{
{\em a} & \% & An element of the ring\\
{\em b} & \% & The right hand side of the equation\\
{\em m} & \% & The nonzero modulus\\
}
\Retval{If the diophantine equation $a x \equiv b \pmod m$ has solutions in
the ring, returns a solution x such that either $x = 0$ or
$\abs x < \abs m$. Returns \failed if the equation has no solution.}
\alpage{divide,quo,rem}
\altarget{divide}
\altarget{quo}
\altarget{rem}
\Usage{divide(a, b)\\a quo b\\ a rem b}
\Signatures{
divide: & (\%,\%) $\to$ (\%, \%)\\
quo,rem: & (\%,\%) $\to$ \%\\
}
\Params{ {\em a,b} & \% & Element of the ring, $y \ne 0$\\ }
\Retval{$a$ rem $b$ returns $r$ such that either $r = 0$
or $0 \le \abs r < \abs b$ and $a \equiv r \pmod b$,
$a$ quo $b$ returns $q$ such that $a - b q = 0$
or $0 \le \abs{a - bq} < \abs b$,
and divide(a, b) returns the pair (a quo b, a rem b).}
\alpage{divide!}
\Usage{\name(x, y, z)}
\Signature{(\%, \%, \%)}{(\%, \%)}
\Params{
{\em x} & \% & An element of the ring (to be destroyed)\\
{\em y} & \% & An element of the ring\\
{\em z} & \% & A placeholder for the quotient (to be destroyed)\\
}
\Retval{Returns $(q, r)$ such that $x = q y + r$ and either $r = 0$
or $0 \le \abs r < \abs x$,
where the storage used by x and z is allowed to be destroyed or reused,
so x and z is lost after this call.}
\Remarks{This function may cause x and z to be destroyed,
so do not use it unless x and z have been locally allocated,
and are guaranteed not to share space
with other elements. Some functions are not necessarily copying their
arguments and can thus create memory aliases.}
\alseealso{\alexp{remainder!}}
\alpage{euclid}
\altarget{\name!}
\Usage{\name(x, y)\\ \name!(x, y)}
\Signature{(\%, \%)}{\%}
\Params{ {\em x,y} & \% & Elements of the ring\\ }
\Retval{Returns $\gcd(x, y)$ computed by the Euclidean algorithm.
When using \name!(x, y), 
the storage used by x and y is allowed
to be destroyed or reused, so x and y are lost after this call.}
\Remarks{The call \name!(x, y) may cause x and y to be destroyed,
so do not use it unless
x and y have been locally allocated, and are guaranteed not to share space
with other elements. Some functions are not necessarily copying their
arguments and can thus create memory aliases.}
\alseealso{\alfunc{GcdDomain}{gcd}, \alfunc{GcdDomain}{gcd!}}
\alpage{euclideanSize}
\Usage{\name~x}
\Signature{\%}{\altype{Integer}}
\Params{ {\em x} & \% & A nonzero element of the ring\\ }
\Retval{Returns $\abs x$, the euclidean size of x. It is connected
to the Euclidean remainder, in that the remainder r of a by b is either 0,
or satisfies $0 \le \abs r < \abs b$.}
\alpage{extendedEuclidean}
\Usage{\name(a, b)\\ \name(a, b, c)}
\Signatures{
\name: & (\%, \%) $\to$ (\%, \%, \%)\\
\name: & (\%, \%, \%) $\to$ \altype{Partial}(\%, \%)\\
}
\Params{ {\em a,b,c} & \% & Elements of the ring\\ }
\Retval{
\name(a, b) returns $(g, x, y)$ such that $g = \gcd(a, b) = a x + b y$.\\
\name(a, b, c) returns either a solution $(x, y)$ of the
diophantine equation $a x + b y = c$, or \failed if it has
no solution in the ring.\\
For the values returned by both calls, either $x = 0$ or
$\abs x < \abs b$.}
\alpage{rationalReconstruction}
\Usage{\name(u, m, n, d)}
\Signature{(\%, \%, \altype{Integer}, \altype{Integer})}
{\altype{Partial} \builtin{Cross}(\%, \%)}
\Params{
\emph{u} & \% & An element of the ring\\
\emph{m} & \% & A nonzero modulus\\
\emph{n,d} & \altype{Integer} & Bounds for the size of the result\\
}
\Retval{Returns either $(a,b)$ such that $a/b = u \pmod m$,
$|a| \le n$ and $|b| \le d$, or \failed~if no such $a,b$ exist.}
\Remarks{
The resulting $a$ and $b$ might not be unique, depending on the
values of the bounds $n$ and $d$.}
\alpage{remainder!}
\Usage{\name(x, y)}
\Signature{(\%, \%)}{\%}
\Params{
{\em x} & \% & An element of the ring (to be destroyed)\\
{\em y} & \% & An element of the ring\\
}
\Retval{Returns the remainder of x by y,
where the storage used by x is allowed
to be destroyed or reused, so x is lost after this call.}
\Remarks{This function may cause x to be destroyed, so do not use it unless
x has been locally allocated, and is guaranteed not to share space
with other elements. Some functions are not necessarily copying their
arguments and can thus create memory aliases.}
