\thistype{PthPowering}
\History{Manuel Bronstein}{22/11/94}{created}
\History{Manuel Bronstein}{7/9/98}{added small characteristic method}
\Usage{import from \this~R}
\Params{
{\em R} & \astype{FiniteCharacteristic} & A finite characteristic ring\\
}
\Descr{\this~provides efficient exponentiation of elements of $R$.}
\begin{exports}
\asexp{pExponentiation}: & (T, \astype{Integer}) $\to$ T & $\sth p$--powering\\
\asexp{pExponentiation!}:
& (T, \astype{Integer}) $\to$ T & In--place $\sth p$--powering\\
\end{exports}
\aspage{pExponentiation}
\astarget{\name!}
\Usage{\name(a, n)\\ \name!(a, n)}
\Signature{(R, \astype{Integer})}{R}
\Params{
{\em a} & R & The element to exponentiate\\
{\em n} & \astype{Integer} & The exponent\\
}
\Retval{Returns $a^n$. The exponent $n$ must be nonnegative.
When using \name!($a, n$),
the storage used by a is allowed
to be destroyed or reused, so a is lost after this call.}
\Remarks{A call to \name!($a, n$) may cause a to be destroyed,
so do not use it unless a has been locally allocated,
and is guaranteed not to share space
with other elements. Some functions are not necessarily copying their
arguments and can thus create memory aliases.}
