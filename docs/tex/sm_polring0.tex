\thistype{PolynomialRing0}
\History{Marc Moreno Maza}{08/07/01}{created}
\Usage{\this~(R,V): Category}
\Params{
{\em R} & \altype{ArithmeticType} & The coefficient domain \\
        & \altype{ExpressionType} &\\
{\em V} & \altype{TotallyOrderedType} & The variable domain \\
         & \altype{ExpressionType} &\\
}
\Descr{\this~(R,V) is the category of the domains that implement
 a polynomial ring with coefficients in $R$ and variables in $V.$
 If $V$ is a finite set $\left\{v_1, \ldots ,v_l \right\}$ 
 this polynomial ring is just
 $R[v_1, \ldots ,v_l].$ 
 If $V$ is finite or not, the set of monomials is
 the free abelian monoid $E$ generated by $V.$ 
 Moreover the default total ordering endowing $E$ is the
 the lexicographical one induced by $V.$
 Observe that the domain $V$ is not assumed to satisfy \altype{VariableType}.
 In fact, only weaker conditions are required.
 Hence it is possible to use any totally ordered set as a domain
 of variables. For instance a set of algebraically independent numbers.
 Observe that \this~(R,V) provides essentially operations
 related to the structure of a free algebra.
 For more sophisticated operations (differentiation, evaluation, \ldots)
 see the category constructor \altype{PolynomialRing}.}
\begin{exports}
\category{\altype{PolynomialTypeRing}(R, V)} \\
\category{\altype{FreeAlgebra} R} \\
\end{exports}
