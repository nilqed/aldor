\thistype{ModularUnivariateGcd}
\History{Laurent Bernardin}{17/8/95}{created}
\History{Manuel Bronstein}{19/4/96}
{Uses optimized CRT function with inverse caching and machine int modulis.
Uses local gcd mod p function rather than instantiating prime fields.
Makes trial divisions only after first try, and when lifted gcd stabilizes.}
\Usage{import from \this(Z, P)}
\Params{
{\em Z} & \altype{IntegerCategory} & An integer-like ring\\
{\em P} & \altype{UnivariatePolynomialAlgebra} Z & Polynomials over Z}
\Descr{\this~provides an implementation of a modular GCD algorithm for
univariate polynomials over the integer, using the Chinese Remainder Theorem.}
\begin{exports}
\alexp{modularGcd}: & (P, P) $\to$ (\altype{Partial} P, P, P) &
The Modular Gcd algorithm\\
\end{exports}
\alpage{modularGcd}
\Usage{\name($p_1, p_2$)}
\Signatures{
\name: & (P, P) $\to$ (\altype{Partial} P, P, P)\\
}
\Params{
{\em $p_1, p_2$} & P & Polynomials over Z\\
}
\Retval{Returns $(g, y, z)$ such that $g = \gcd(p_1, p_2)$ or \failed,
and if $g$ is not \failed, then $p = y g$ and $q = z g$.}
\Remarks{This algorithm can fail because it runs out of primes. This will happen
if the gcd has coefficients with more than around 3000 digits.}
