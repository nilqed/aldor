\thistype{PrimeCollection}
\History{Manuel Bronstein}{26/4/96}{created}
\Usage{\this: Category}
\Descr{\this~is the category of collections of primes with various
properties and various sizes.}
\begin{exports}
\alexp{allPrimes}: & () $\to$ \altype{Generator} Z & Generate all the primes\\
\alexp{fourierPrime}: & Z $\to$ (Z, Z) & Fourier prime\\
\alexp{maxPrime}: & $\to$ Z & Largest prime\\
\alexp{nextPrime:} & Z $\to$ Z & First prime above a given number\\
\alexp{previousPrime:} & Z $\to$ Z & First prime below a given number\\
\alexp{primeInCollection?:} & Z $\to$ \altype{Boolean} & Check a prime\\
\alexp{primRoot:} & Z $\to$ Z & Modular primitive root\\
\alexp{randomPrime:} & () $\to$ Z & Random prime\\
\end{exports}
\begin{alwhere}
Z & == & \altype{MachineInteger}\\
\end{alwhere}
\alpage{allPrimes}
\Usage{ for p in \name() repeat \{ \dots \} }
\Signature{()}{\altype{Generator} \altype{MachineInteger}}
\Descr{This function allows a loop to iterate over all the primes
provided by the collection.}
\alpage{fourierPrime}
\Usage{\name~n}
\Signature{\altype{MachineInteger}}
{(\altype{MachineInteger},\altype{MachineInteger})}
\Retval{Returns $(p,\omega)$ such that $p$ is a prime of the form
$p = 2^n k + 1$ with $k$ odd, and $\omega$ is a primitive $\sth{2^n}$
root of unity in $\mathbbm{F}_p$. Returns $(0,0)$ if $n$ is too large.}
\alpage{maxPrime}
\Usage{\name}
\alconstant{\altype{MachineInteger}}
\Retval{Returns the largest prime in the collection.}
\alpage{nextPrime}
\Usage{\name~n}
\Signature{\altype{MachineInteger}}{\altype{MachineInteger}}
\Params{ {\em n} & \altype{MachineInteger} & An integer\\ }
\Retval{Returns the smallest prime $p$ in the collection with $n < p$,
$0$ if there are none.}
\alseealso{previousPrime(\this), randomPrime(\this)}
\alpage{previousPrime}
\Usage{\name~n}
\Signature{\altype{MachineInteger}}{\altype{MachineInteger}}
\Params{ {\em n} & \altype{MachineInteger} & An integer\\ }
\Retval{Returns the largest prime $p$ in the collection with $n > p$,
$0$ if there are none.}
\alseealso{nextPrime(\this), randomPrime(\this)}
\alpage{primeInCollection?}
\Usage{\name~n}
\Signature{\altype{MachineInteger}}{\altype{Boolean}}
\Params{ {\em n} & \altype{MachineInteger} & An integer\\ }
\Retval{Returns \true{} if \emph{n} is a prime in the collection,
\false{} otherwise (\emph{n} could still be prime in that case).}
\alpage{primRoot}
\Usage{\name~p}
\Signature{\altype{MachineInteger}}{\altype{MachineInteger}}
\Params{ {\em p} & \altype{MachineInteger} & A prime\\ }
\Retval{Returns a generator of the multiplicative group
$(\ZZ/p \ZZ)^\ast$ or $0$ if $p$ is not a prime in the table,
or is no primitive root is stored.}
\alseealso{primitiveRoot(\altype{PrimitiveRoots})}
\alpage{randomPrime}
\Usage{\name()}
\Signature{()}{\altype{MachineInteger}}
\Retval{Returns a random prime in the collection.}
\alseealso{nextPrime(\this), previousPrime(\this)}
