\thistype{UnivariatePolynomialQuotient}
\History{Manuel Bronstein}{27/12/96}{created}
\History{Manuel Bronstein}{30/6/2003}{split UnivariatePolynomialQuotientSqfr}
\History{Manuel Bronstein}{30/6/2003}{added newtonSeries}
\Usage{ \this(R, Rx):Category }
\Params{
{\em R} & \altype{CommutativeRing} & The coefficient ring of the polynomials\\
{\em Rx} & \altype{UnivariatePolynomialAlgebra} R & A polynomial type over R\\
}
\Descr{\this(R, Rx) is the category of extensions of $R$ of the form
$Rx / (p)$ for some $p \in Rx$, where \emph{p} is assumed to be monic,
but with no other restrictions.
Use instead the categories \altype{UnivariatePolynomialQuotientSqfr}
when \emph{p} is also known to
be squarefree, and \altype{SimpleAlgebraicExtensionCategory}
when \emph{p} is also known to be irreducible.}
\begin{exports}
\category{\altype{Algebra} R}\\
\category{\altype{CommutativeRing}}\\
\alfunc{IndexedFreeLinearCombinationType}{coefficient}:
& (\%, Z) $\to$ R & Extraction of a coefficient\\
\alexp{compose}: & \% $\to$ \% $\to$ \% & Modular composition\\
\alexp{definingPolynomial}: & Rx & Defining polynomial\\
\alfunc{IndexedFreeModule}{generator}:
& \% $\to$ \altype{Generator} \albuiltin{Cross}(R, Z) &
iterate through all the terms\\
\alexp{knownIrreducible?}: & \altype{Boolean} & Irreducible modulus?\\
\alexp{lift}: & \% $\to$ Rx & Conversion to a polynomial\\
\alfunc{FreeLinearCombinationType}{map}:
& (R $\to$ R) $\to$ \% $\to$ \% & Lift a mapping\\
\alfunc{FreeLinearCombinationType}{map!}:
& (R $\to$ R) $\to$ \% $\to$ \% & Lift a mapping\\
\alexp{monom}: & \% & Generator of the algebra\\
\alexp{reduce}: & Rx $\to$ \% & Reduction of a polynomial\\
\alexp{value}:
& (P:POLY \%) $\to$ (P, Z, Z) $\to$ \% & Evaluation at a rational number\\
\end{exports}
\begin{alwhere}
Z &==& \altype{Integer}\\
POLY &==& \altype{UnivariatePolynomialAlgebra}\\
\end{alwhere}
\begin{exports}[if R has \altype{CharacteristicZero} then]
\category{\altype{CharacteristicZero}}\\
\end{exports}
\begin{exports}[if R has \altype{FiniteCharacteristic} then]
\category{\altype{FiniteCharacteristic}}\\
\end{exports}
\begin{exports}[if R has \altype{FiniteSet} then]
\category{\altype{FiniteSet}}\\
\end{exports}
\alpage{compose}
\Usage{\name(p)(q)}
\Params{ {\em p, q} & \% & Polynomials\\ }
\Retval{Returns
$$
q(p) = \sum_{i=0}^n a_i p^i
$$
where $q = \sum_{i=0}^n a_i x^i$.}
\Remarks{If you want to compute
$q_1(p),\dots,q_k(p)$ for several $q_i$'s, use the curried version
as follows: {\tt f := compose p; for i in 1..k repeat r.i := f(q.i); },
since the various calls to {\tt f} will share a table of powers of $p$.}
\alpage{definingPolynomial}
\Usage{\name}
\alconstant{Rx}
\Retval{Returns the polynomial $p$ such that the extension is $Rx/(p)$.}
\alpage{knownIrreducible?}
\Usage{\name}
\alconstant{\altype{Boolean}}
\Retval{Returns \true{} if the modulus is known to be irreducible,
\false{} otherwise. Note that the modulus could be irreducible, even
if \name{} is \false.}
\alpage{lift}
\Usage{\name~q}
\Signature{\%}{Rx}
\Params{ {\em q} & \% & An element of the algebraic extension\\ }
\Retval{Returns $q$ as an element of $Rx$.}
\alpage{monom}
\Usage{\name}
\alconstant{\%}
\Retval{Returns the image in the quotient of the term \emph{x} from \emph{Rx}.
That element generates this type as an algebra over \emph{R}.}
\alseealso{\alfunc{UnivariateFreeLinearArithmeticType}{monom}}
\alpage{reduce}
\Usage{\name~q}
\Signature{Rx}{\%}
\Params{ {\em q} & Rx & A polynomial\\ }
\Retval{Returns the remainder of $q$ modulo $p$ as an element of $Rx / (p)$.}
\alpage{value}
\Usage{\name(P)(p, n, d)}
\Signature{(P:\altype{UnivariatePolynomialAlgebra} \%)}
{(P, \altype{Integer}, \altype{Integer}) $\to$ \%}
\Params{
{\em P} & \altype{UnivariatePolynomialAlgebra} \% & A polynomial type\\
{\em p} & P & A polynomial\\
{\em n} & \altype{Integer} & A numerator\\
{\em d} & \altype{Integer} & A nonzero denominator\\
}
\Retval{Returns $d^e p(n/d)$ where $e$ is the smallest nonnegative exponent
such that $d^e p(n/d)$ is an element of the extension.}
\thistype{UnivariatePolynomialQuotientSqfr}
\History{Manuel Bronstein}{30/6/2003}{created}
\Usage{ \this(R, Rx):Category }
\Params{
{\em R} & \altype{CommutativeRing} & The coefficient ring of the polynomials\\
{\em Rx} & \altype{UnivariatePolynomialAlgebra} R & The type of the modulus\\
}
\Descr{\this(R, Rx) is the category of extensions of $R$ of the form
$Rx / (p)$ for some $p \in Rx$, where \emph{p} is assumed to be monic
and squarefree, but not not necessarily irreducible.
Use instead the categories \altype{UnivariatePolynomialQuotient}
when \emph{p} is not squarefree,
and \altype{SimpleAlgebraicExtensionCategory}
when \emph{p} is known to be irreducible.}
\begin{exports}
\category{\altype{UnivariatePolynomialQuotient}(R, Rx)}\\
\alexp{newtonSeries}: & Rxx & Newton series of the generator\\
\alexp{norm}: & \% $\to$ R & Norm\\
              & (P:POLY \%) $\to$ P $\to$ Rx & \\
\alexp{trace}: & \% $\to$ R & Trace\\
               & (P:POLY \%) $\to$ P $\to$ Rx & \\
\end{exports}
\begin{alwhere}
POLY &==& \altype{UnivariatePolynomialAlgebra}\\
Rxx &==& \altype{DenseUnivariateTaylorSeries} R\\
\end{alwhere}
\begin{exports}[if R has \altype{RationalRootRing} then]
\category{\altype{RationalRootRing}}\\
\end{exports}
\begin{exports}
[if R has \altype{CharacteristicZero} and R has \altype{Field} then]
\category{\altype{DifferentialExtension} R}\\
\end{exports}
\alpage{newtonSeries}
\Usage{\name}
\alconstant{\altype{DenseUnivariateTaylorSeries} R}
\Retval{Returns the series
$$
\sum_{n\ge 0} Tr(\alpha^n) x^n
$$
where $\alpha$ is the image in the quotient of the term \emph{x} from \emph{Rx},
and $Tr$ is the trace from the quotient into \emph{R}.}
\alseealso{\alfunc{UnivariatePolynomialQuotient}{monom},\alexp{trace}}
\alpage{norm,trace}
\altarget{norm}
\altarget{trace}
\Usage{norm~q\\trace~q\\norm~P\\trace~P\\norm(P)(p)\\trace(P)(p)}
\Signatures{
\name: & \% $\to$ R\\
\name: & (P:\altype{UnivariatePolynomialAlgebra} \%) $\to$ P $\to$ Rx\\
}
\Params{
{\em q} & \% & An element of the algebraic extension\\
{\em P} & \altype{UnivariatePolynomialAlgebra} \% & A polynomial type\\
{\em p} & P & A polynomial\\
}
\Descr{norm($q(\alpha)$) and trace($q(\alpha)$) return respectively
the product and sum of the $q(\alpha)$ over all the roots
of the polynomial defining the extension,
while norm(P)($p(\alpha,x)$) and trace(P)($p(\alpha,x)$) return respectively
the product and sum of the $p(\alpha,x)$ over all the roots
of the polynomial defining the extension.}
