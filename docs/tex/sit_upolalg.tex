\thistype{UnivariatePolynomialRing}
\History{Manuel Bronstein}{16/6/2000}{separated from sit_uffalg.as}
\Usage{\this~R: Category}
\Params{
{\em R} & \altype{ExpressionType} & The coefficient domain\\
        & \altype{ArithmeticType} &\\
}
\Descr{\this~is a common category for commutative and noncommutative
univariate polynomials with coefficients in an arbitrary arithmetic
system R and with respect to the power basis $(x^n)_{n \ge 0}$.}
\begin{exports}
\category{\altype{UnivariateFreeRing} R}\\
\alexp{add!}: & (\%,R,Z,\%, Z, Z) $\to$ \% & In-place partial product and sum\\
\alexp{compose}: & (\%, \%) $\to$ \% & Compose polynomials\\
\alexp{translate}: & (\%, R) $\to$ \% & Translate a polynomial\\
\end{exports}
\begin{exports}[if $R$ has \altype{Parsable} then]
\category{\altype{Parsable}}\\
\end{exports}
\begin{aswhere}
Z &==& \altype{Integer}\\
\end{aswhere}
\alpage{add!}
\Usage{\name(p, c, m, q, n, N)}
\Signature{(\%,R,\altype{Integer},\%,\altype{Integer},\altype{Integer})}{\%}
\Params{
{\em p} & \% & A polynomial (to be destroyed)\\
{\em c} & R & A scalar\\
{\em m} & \altype{Integer} & The degree of the monomial to add\\
{\em q} & \% & A polynomial to be multiplied by $c x^m$ and added to p\\
{\em n} & \altype{Integer} & A lower threshold\\
{\em N} & \altype{Integer} & An upper treshold\\
}
\Descr{\name(p, c, m, q, n, N) computes all the terms of degree at least $n$
and at most $N$ of
$$
p + c x^m q = \sum_{i=0}^{d+m} (a_i + c b_{i-m}) x^i\,,
$$
where $p = \sum_{i=0}^d a_i x^i$ and $q = \sum_{i=0}^d b_i x^i$.
Note that $m$ is allowed to be negative.
For efficiency reasons it is sometimes sufficient to compute some terms of
that sum only. All other coefficients of $p$ are not changed.}
\Remarks{The storage used by p is allowed to be destroyed or reused, so p
is lost after this call. This may cause p to be destroyed, so do not use
this unless p has been locally allocated, and is thus guaranteed not to
share space with other polynomials. Some functions, like \alexp{reductum} are
not necessarily copying their arguments and can thus create memory aliases.}
\alpage{compose,translate}
\altarget{compose}
\altarget{translate}
\Usage{compose(p, q)\\ translate(p, r)}
\Params{
{\em p, q} & \% & Polynomials\\
{\em r} & R & Amount to translate\\
}
\Retval{compose(p, q) returns
$$
p(q) = \sum_{i=0}^n a_i q^i
$$
where $p = \sum_{i=0}^n a_i x^i$, while translate(p, r) returns $p(x - r)$.}
