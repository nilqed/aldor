\thistype{Permutation}
\History{Manuel Bronstein}{9/12/1999}{created}
\History{Manuel Bronstein}{30/01/2003}{bug fix: transpose! does not trash 1}
\Usage{import from \this~n}
\Params{{\em n} & \altype{MachineInteger} & The number of elements\\}
\Descr{\this(n) implements the group of permutations on $n$ elements.}
\begin{exports}
\category{\altype{CopyableType}}\\
\category{\altype{Group}}\\
\alexp{apply}: & (\%, Z) $\to$ Z & Image of an element\\
\alexp{mapping}: & \% $\to$ (Z $\to$ Z) & Action of a permutation\\
\alexp{sign}: & \% $\to$ Z & Sign\\
\alexp{transpose}: & (Z, Z) $\to$ \% & Transposition\\
\alexp{transpose!}: & (\%, Z, Z) $\to$ \% & Compose with a transposition\\
\end{exports}
\begin{alwhere}
Z &==& \altype{MachineInteger}\\
\end{alwhere}
\alpage{apply}
\Usage{ \name($\sigma$, x) \\ $\sigma x$}
\Signature{(\%, \altype{MachineInteger})}{\altype{MachineInteger}}
\Params{
{\em $\sigma$} & \% & A permutation\\
{\em x} & \altype{MachineInteger} & An index\\
}
\Retval{Returns $\sigma x$.}
\alpage{mapping}
\Usage{\name~$\sigma$}
\Signature{\%}{(\altype{MachineInteger} $\to$ \altype{MachineInteger})}
\Params{ {\em $\sigma$} & \% & A permutation\\ }
\Retval{Returns the map corresponding to $\sigma$.}
\alseealso{\alexp{apply}}
\alpage{sign}
\Usage{\name~$\sigma$}
\Signature{\%}{\altype{MachineInteger}}
\Params{ {\em $\sigma$} & \% & A permutation\\ }
\Retval{Returns the sign of $\sigma$, \ie $(-1)^\epsilon$
where $\epsilon$ is the number of tranpositions in the factorization
of $\sigma$.}
\alpage{transpose}
\altarget{\name!}
\Usage{\name(x,y)\\ \name!($\sigma$,x,y)}
\Signatures{
\name: & (\altype{MachineInteger}, \altype{MachineInteger}) $\to$ \%\\
\name!: & (\%, \altype{MachineInteger}, \altype{MachineInteger}) $\to$ \%\\
}
\Params{
{\em $\sigma$} & \% & A permutation\\
{\em x,y} & \altype{MachineInteger} & Indices\\
}
\Retval{\name(x,y) returns the transposition $(x y)$, while
\name!($\sigma$,x,y) returns the composition $(x y) \circ \sigma$.}
\Remarks{When using \name!, the storage used by $\sigma$ is allowed to
be destroyed or reused, so do not use it unless $\sigma$ has been
locally allocated, and is guaranteed not to share space
with other elements. Some functions are not necessarily copying their
arguments and can thus create memory aliases. Since there is no
guarantee of reuse, you should always use the permutation returned
by \name! rather than $\sigma$ after the call.}
